\begin{theo}[Magnetische velden]{Magnetische velden}
    Net zoals we rondom een elektrische lading een elektrisch veld hebben gedefinieerd, kunnen we rond een magneet een magnetisch veld
    $\Vec{B}$ definiëren. Magnetische veldlijnen op tekeningen hebben dusdanig ook dezelfde eigenschappen als elektrische veldlijnen, namelijk

    \begin{minipage}{.7\textwidth}
        \begin{itemize}
            \item de richting van het magnetische veld is tangentieel met de magnetische veldlijnen
            \item de hoeveelheid magnetische veldlijnen per oppervlakte duidt op het sterkte van het magnetische veld
        \end{itemize}

    \end{minipage}
    \begin{minipage}{.24\textwidth}
        \includegraphics[scale = 0.5]{Images/Magnetisme/HomogeenMagnetischVeld}
    \end{minipage}

    \noindent De bijlage toont het triviale voorbeeld van een homogeen magnetisch veld.
\end{theo}

\begin{app}[Elekrtische stroom]{Elekrtische stroom}
    Elektrische stroom \textbf{produceert} een magnetisch veld. Een niet-magnetische, geleidende draad is dus wel magnetisch als we er stroom op zetten.
    Om de richting van het magnetisch veld te weten, kunnen we de \textbf{rechterhandregel} toepassen:

    \begin{minipage}{.8\textwidth}
        pak de draad vast met je rechterhand en steek je duim uit in
        de richting van de conventionele stroomzin en vouw je vingers dicht in de richting van het magnetische veld.
    \end{minipage}
    \begin{minipage}{.16\textwidth}
        \includegraphics[scale = 0.2]{Images/Magnetisme/Rechterhandregel}
    \end{minipage}
\end{app}

\begin{theo}[Magnetische kracht]{Magnetische kracht}
    De relatie tussen de magnetische kracht $\Vec{F}$ van een draad met stroom $I$ en het magnetisch veld $\Vec{B}$ kan geschreven worden als een vector product
    \begin{equation*}
        d\Vec{F}= Id\Vec{\ell} \times \Vec{B}
    \end{equation*}
    waarbij $d\Vec{F}$ een infinitesimale kracht op een infinitesimale lengte $d\Vec{\ell}$ van de draad in het magnetische veld.
    \begin{center}
        \includegraphics[scale = 0.4]{Images/Magnetisme/MagnetischeKracht}
    \end{center}
\end{theo}

\newpage

\begin{app}[Magnetische kracht in een homogeen veld]{Magnetische kracht in een homogeen veld}
    \vspace{-0.5cm}
    \begin{minipage}{.75\textwidth}
        Als het veld homogeen is, dan zal elk infinitesimaal deeltje $d\Vec{\ell}$ dezelfde hoek maken met het magnetische veld $\Vec{B}$.
        Hieruit volgt dan de formule:
        \begin{equation*}
            \Vec{F} = I\Vec{\ell} \times \Vec{B}
        \end{equation*}
    \end{minipage}
    \begin{minipage}{.21\textwidth}
        \includegraphics[scale = 0.25]{Images/Magnetisme/DraadHomogeenVeld}
    \end{minipage}
\end{app}

%\begin{ex}[Willekeurige stroomvoerende geleider in een homogeen magnetisch veld]{Willekeurige stroomvoerende geleider in een homogeen magnetisch veld}
%    \begin{minipage}{.6\textwidth}
%        \vspace{-0.5cm}
%        \begin{align*}
%            \Vec{F} &= I\int_a^{b}d\Vec{l} \times \Vec{B} \\
%            &= I(\int_a^{b}d\Vec{l} \times \Vec{B}) \quad \text{(B is homogeen)} \\
%            &= I(\Vec{L} \times \Vec{B})
%        \end{align*}
%    \end{minipage}
%    \begin{minipage}{.36\textwidth}
%        \includegraphics[scale = 0.35]{Images/Magnetisme/StroomvoerendeGeleiderHomogeenMagnetischVeld}
%    \end{minipage}
%\end{ex}

\begin{theo}[Magnetische kracht op een bewegende lading]{Magnetische kracht op een bewegende lading}
    Elektrische stroom is een verzameling van N bewegende ladingen, we kunnen de definitie van magnetische kracht dus ook anders schrijven:
    \begin{equation*}
        \Vec{F} = I\Vec{\ell} \times \Vec{B} = Nq\Vec{v} \times \Vec{B}
    \end{equation*}
    sinds $I = \tfrac{Nq}{t}$ en $\Vec{\ell} = \Vec{v}t$.
\end{theo}

\begin{app}[Bewegende lading in een vlak loodrecht op magnetisch veld]{Bewegende lading in een vlak loodrecht op magnetisch veld}
    \begin{minipage}{.67\textwidth}
        Het pad van een bewegende lading in een vlak loodrecht op het magnetisch veld is cirkelvormig.
        De kracht zal altijd loodrecht zijn op de snelheid, dus de lading zal cirkelvormig bewegen met
        centripetale versnelling:
        \begin{equation*}
            a = \dfrac{v^2}{r}.
        \end{equation*}
        De periode heeft de volgende formule:
        \begin{equation*}
            T = \dfrac{2\pi r}{v} = \dfrac{2\pi m}{qB} = \dfrac{1}{f}
        \end{equation*}
        sinds $r = \tfrac{mv}{qB}$.
    \end{minipage}
    \begin{minipage}{.29\textwidth}
        \includegraphics[scale = 0.35]{Images/Magnetisme/CirkelvormigeBewegingMagentischVeld}
    \end{minipage}
\end{app}