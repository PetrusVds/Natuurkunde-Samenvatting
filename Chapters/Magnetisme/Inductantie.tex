\begin{theo}[Wederzijdse inductie]{Wederzijdse inductie}
    \begin{minipage}{.78\textwidth}
        Als twee spoelen nabij elkaar plaatst worden, zoals in de figuur, dan zal een veranderende stroom in de ene, 
        een emf induceren in de andere. Dus: geïnduceerde emf in een spoel is evenredig met de snelheid van de stroomverandering
        in de andere. Dit noemt men \textbf{wederzijdse inductie} $M$, we noteren
        \begin{equation*}
            M_{21} = \dfrac{N_{2}\Phi_{21}}{I_{1}} 
        \end{equation*}
        met $M_{21}$ de wederzijdse inductiecoëfficiënt en $\Phi_{21}$ de magnetische flux doorheen spoel 2 tegenover de stroom in spoel 1.
        We kunnen dit mengen met de wet van Faraday
    \end{minipage}
    \begin{minipage}{.18\textwidth}
        \includegraphics[scale=0.3]{Images/Magnetisme/WederzijdseInductie}
    \end{minipage}
    \begin{equation*}
        \mathcal{E}_{2} = - N_{2}\dfrac{d\Phi_{21}}{dt} = - M_{21}\dfrac{dI_{1}}{dt}
    \end{equation*}
    waarbij nu de verandering in stroom in spoel 1 hebben verbonden aan de emf dat het induceert in spoel 2.  In de algemene situatie is 
    $M = M_{12} = M_{21}$.
\end{theo}

% \begin{theo}[Wederzijdse inductie]{Wederzijdse inductie}
%     \begin{minipage}{.78\textwidth}
%         Als twee spoelen dicht bij elkaar worden geplaatst, zoals in de figuur, zal een veranderende stroom in de ene spoel een elektromotorische kracht (emf) induceren in de andere spoel. Met andere woorden, de geïnduceerde emf in een spoel is evenredig met de snelheid van de stroomverandering in de andere spoel. Dit staat bekend als \textbf{wederzijdse inductie} $M$, waarbij we noteren:
%         \begin{equation*}
%             M_{21} = \dfrac{N_{2}\Phi_{21}}{I_{1}} 
%         \end{equation*}
%         Hierbij staat $M_{21}$ voor de wederzijdse inductiecoëfficiënt en $\Phi_{21}$ voor de magnetische flux door spoel 2 als gevolg van de stroom in spoel 1.
%         Dit kan worden gecombineerd met de wet van Faraday.
%     \end{minipage}
%     \begin{minipage}{.18\textwidth}
%         \includegraphics[scale=0.3]{Images/Magnetisme/WederzijdseInductie}
%     \end{minipage}
%     \begin{equation*}
%         \mathcal{E}_{2} = - N_{2}\dfrac{d\Phi_{21}}{dt} = - M_{21}\dfrac{dI_{1}}{dt}
%     \end{equation*}
%     Hiermee verbinden we de verandering in stroom in spoel 1 met de emf die het induceert in spoel 2. In de algemene situatie is $M = M_{12} = M_{21}$.
% \end{theo}


% \begin{app}[Wederzijdse inductie van een solenoïde en een spoel]{Wederzijdse inductie van een solenoïde en een spoel}
%     De solenoïde is opeengepakt, dus kunnen we veronderstellen dat alle magnetische flux in de solenoïde blijft in de tweede spoel.
%     We weten dam wat de flux door deze spoel is, namelijk
%     \begin{equation*}
%         \Phi_{21} = BA = \mu_{0}\dfrac{N_{1}}{\ell}I_{1}A
%     \end{equation*}
%     en de wederzijdse inductiecoëfficient
%     \begin{equation*}
%         M = \dfrac{N_{2}\Phi_{21}}{I_{1}} = \mu_{0}\dfrac{N_{1}N_{2}}{\ell}A
%     \end{equation*}
%     waarbij we zien dat deze enkel afhangt van de geometrie van het systeem.
% \end{app}

\begin{app}[Wederzijdse inductie van een solenoïde en een spoel]{Wederzijdse inductie van een solenoïde en een spoel}
    De solenoïde is compact, dus kunnen we aannemen dat alle magnetische flux in de solenoïde blijft en door de tweede spoel gaat.
    We weten dan wat de flux door deze spoel is, namelijk
    \begin{equation*}
        \Phi_{21} = BA = \mu_{0}\dfrac{N_{1}}{\ell}I_{1}A
    \end{equation*}
    en de wederzijdse inductiecoëfficiënt is
    \begin{equation*}
        M = \dfrac{N_{2}\Phi_{21}}{I_{1}} = \mu_{0}\dfrac{N_{1}N_{2}}{\ell}A
    \end{equation*}
    waaruit blijkt dat deze alleen afhankelijk is van de geometrie van het systeem.
\end{app}


\begin{theo}[Zelfinductie]{Zelfinductie}
    De magnetische flux $\Phi_{B}$ door een spoel met $N$ windingen (of eenderwelke kring) is evenredig met de stroom $I$ die erdoor vloeit, dus we definiëren de \textbf{zelfinductie} $L$ als
    \begin{equation*}
        L = \dfrac{N\Phi_{B}}{I}
    \end{equation*}
    Dus de geïnduceerde emf $\mathcal{E}$ in een spoel met zelfinductie $L$ is, volgens de wet van Faraday, gelijk aan
    \begin{equation*}
        \mathcal{E} = -N\dfrac{d\Phi_{B}}{dt }= - L\dfrac{dI}{dt}
    \end{equation*}
    \vspace{-0.3cm}
\end{theo}

\newpage

\begin{app}[Energie opgeslagen in een magnetisch veld]{Energie opgeslagen in een magnetisch veld}
    Wanneer een inductor met zelfinductie $L$ een stroom $I$ heeft, die verandert met een snelheid $dI/dt$, dan is de energie die geleverd wordt aan de inductor gelijk aan 
    \begin{equation*}
        P_L = LI\dfrac{dI}{dt}
    \end{equation*}
    De hoeveelheid arbeid $dW$ geleverd aan de inductor in een tijd $dt$ is gelijk aan
    \begin{equation*}
        dW_L = P_Ldt = LIdI
    \end{equation*}
    De energie die opgeslagen is in het magnetisch veld van de inductor is gelijk aan de energie die geleverd is aan de inductor
    \begin{equation*}
        U_L = \int dW_L = \int P_Ldt = \int_{0}^{I} -LIdI = \dfrac{1}{2}LI^{2}
    \end{equation*}
    wat heel vergelijkbaar is aan de formule voor de energie opgeslagen in een condensator, namelijk: $U_{C} = \frac{1}{2}C(\Delta V)^{2}$. 
    Net zoals de energie in een condensator zich bevindt in het elektrisch veld tussen de platen,bevindt de energie in een inductor zich in het magnetisch veld binnenin de spoel.
\end{app}

\begin{vrg}[Energiedichtheid]{Energiedichtheid}   
    \vspace{-0.3cm}
    \def\arraystretch{2}
    \centering
    \begin{tabular}{c|c}
        Energiedichtheid van een elektrisch veld & Energiedichtheid van een magnetisch veld \\[-0.35cm]
        van een spoel met zelfinductie $L$ en stroom $I$ & van een condensator met capaciteit $C$ en spanning $V$ \\ \hline
        $u_{E} = \dfrac{1}{2}\epsilon_{0}E^{2}$ & $u_{B} = \dfrac{1}{2}\mu_{0}B^{2}$ \\
    \end{tabular}
\end{vrg}

\begin{app}[RL-kringen]{RL-kringen}
    \vspace{-0.5cm}
    \begin{minipage}{.66\textwidth}
        We sluiten de schakelaar en we passen de tweede wet van Kirchhoff toe op de kring. We krijgen
        \begin{equation*}
            V_{0} - IR - \mathcal{E} = V_{0} - IR - L\dfrac{dI}{dt} = 0
        \end{equation*}
        We kunnen dit herwerken tot een differentiaalvergelijking
        \begin{equation*}
            L\dfrac{dI}{dt} + RI = V_{0}
        \end{equation*}
        wat we oplossen tot 
        \begin{equation*}
            I(t) = \dfrac{V_{0}}{R}\left(1 - e^{\tfrac{-t}{\tau}}\right)
        \end{equation*} 
    \end{minipage}
    \begin{minipage}{.3\textwidth}
        \hspace{0.5cm}\includegraphics[scale = 0.35]{Images/Magnetisme/RLKring}
        \includegraphics[scale = 0.35]{Images/Magnetisme/RLKringGraph.png}
    \end{minipage}

    \noindent met $\tau = \tfrac{L}{R}$. We zien dat de stroom initieel snel stijgt en dan afvlakt geleidelijk nabij $\tfrac{\mathcal{E}}{R}$.
\end{app}

\begin{app}[LC-kringen]{LC-kringen}
    \vspace{-0.2cm}
    \begin{minipage}{.73\textwidth}
        We bekijken nu een kring met een inductor met inductantie $L$ en een condensator met capaciteit $C$; dit is dus een ideale kring zonder weerstand. 
        We gaan ervan uit dat er initieel lading is in de condensator en er dus een potentiaal verschil $\Delta V = \tfrac{Q}{C}$ op is. Stel dat we de schakelaar sluiten, 
        dan zal de condensator beginnen te ontladen en zal de stroom door de inductor beginnen te stijgen. Hierop passen we de tweede wet van Kirchhoff toe:
    \end{minipage}
    \begin{minipage}{.23\textwidth}
        \vspace{0.3cm}\includegraphics[scale = 0.35]{Images/Magnetisme/LCKring}
    \end{minipage}
    \vspace{-0.3cm}
    \begin{equation*}
        -L\dfrac{dI}{dt} + \dfrac{Q}{C} = 0.
    \end{equation*}
    De lading op de positieve plaat, die voor de stroom zorgt, zal verminderen en dus ook de stroom. We hebben nu $I = -\tfrac{dQ}{dt}$ en we kunnen nu de bovenstaande vergelijking herwerken tot
    een differentiaalvergelijking
    \begin{equation*}
        \dfrac{d^{2}Q}{dt^{2}} + \dfrac{Q}{LC} = 0
    \end{equation*}
    wat dezelfde vorm heeft als de vergelijking van de harmonische beweging. De oplossing is dus

    \begin{minipage}{.71\textwidth}
        \begin{align*}
            Q(t) &= Q_{\max}\cos(\omega t + \phi) \\
                 &= Q_{\max}\cos(\tfrac{t}{\sqrt{LC}} + \phi)
        \end{align*}
        \hspace{-0.6cm} en voor de stroom
        \begin{align*}
            I(t) &= -\dfrac{dQ(t)}{dt} \\
                 &= \dfrac{Q_{\max}}{\sqrt{LC}}\sin(\tfrac{t}{\sqrt{LC}} + \phi) \\
                 &= I_{\max}\sin(\tfrac{t}{\sqrt{LC}} + \phi)
        \end{align*}
        \vspace{0.1cm}
    \end{minipage}
    \begin{minipage}{.25\textwidth}
        \vspace{-0.3cm}\hspace{-0.45cm}\includegraphics[scale = 0.45]{Images/Magnetisme/LCKringGrafiek}
    \end{minipage}

    \vspace{-0.4cm}

    \noindent met $\phi$ de faseconstante. We zien dat de lading en de stroom sinusvormige functies zijn van de tijd. De stroom is $90^{\circ}$ uit fase met de lading en de frequentie van de oscillaties is
    \begin{equation*}
        f = \dfrac{1}{2\pi\sqrt{LC}}
    \end{equation*}
    waaruit we $\omega = 2\pi f = \tfrac{1}{\sqrt{LC}}$ kunnen afleiden. De energie in het elektrisch veld van de condensator is

    \begin{minipage}{.70\textwidth}
        \begin{equation*}
            U_{E} = \dfrac{Q^{2}}{2C} = \dfrac{Q_{\max}^2}{2C}\cos^2(\omega t + \phi)
        \end{equation*}
        \hspace{-0.6cm} en de energie in het magnetische veld van de inductor is 
        \begin{equation*}
            U_{B} = \dfrac{LI^{2}}{2} = \dfrac{Q_{\max}^2}{2C}\sin^2(\omega t + \phi).
        \end{equation*}
    \end{minipage}
    \begin{minipage}{.26\textwidth}
        \vspace{0cm}\hspace{-1.4cm}\includegraphics[scale = 0.4]{Images/Magnetisme/LCKringOscillatie.png} 
    \end{minipage}
    De totale energie in de kring is dus
    \begin{align*}
        U &= U_{E} + U_{B} \\
          &= \dfrac{Q_{\max}^2}{2C}\left(\cos^2(\omega t + \phi) +\sin^2(\omega t + \phi)\right) \\ 
          &= \dfrac{Q_{\max}^2}{2C}
    \end{align*}
    \vspace{-0.3cm}
\end{app}

\newpage

\begin{app}[RLC-kringen]{RLC-kringen}
    \vspace{-0.5cm}
    \begin{minipage}{.73\textwidth}
        \vspace{0.2cm}
        We bekijken een kring met twee schakelaars, een weerstand $R$, een condensator met capaciteit $C$ en een inductor
        met inductantie $L$. We sluiten schakelaar 1 (zoals op de figuur) opdat er lading zou zijn op de condensator. 
        Vervolgens openen we deze weer en sluiten we de tweede schakelaar. Er is nu een LC-oscillatie waarbij er energie verloren gaat aan de weerstand.
        We passen de tweede wet van Kirchhoff toe op de kring
    \end{minipage}
    \begin{minipage}{.23\textwidth}
        \vspace{0.3cm}\includegraphics[scale = 0.45]{Images/Magnetisme/RLCKring}
    \end{minipage}
    \vspace{-0.25cm}
    \begin{equation*}
        - LI\frac{dI}{dt} - IR + Q/C = 0
    \end{equation*}
    wat we kunnen omvormen tot
    \begin{equation*}
        L\frac{d^2Q}{dt^2} + R\dfrac{dQ}{dt} + \dfrac{Q}{C} = 0
    \end{equation*}
    sinds $I = -\frac{dQ}{dt}$. De oplossing van deze differentiaalvergelijking is
    \begin{equation*}
        Q(t) = Q_{\max}e^{\frac{-R}{2L}t}\cos(\omega t + \phi)
    \end{equation*}
    met $\omega = \sqrt{\frac{1}{LC} - \left(\frac{R}{2L}^2\right)}$ en $\phi$ de faseconstante. We zien dus duidelijk als R groot is,
    dat we kunnen spreken van \textbf{overdemping}.
\end{app}

\begin{theo}[Wisselstroomkringen]{Wisselstroomkringen}
    Wisselstroomkringen zijn kringen waarbij de stroom varieert in de tijd. De stroom kan worden beschreven door
    \begin{equation*}
        I(t) = I_{\max}\cos(\omega t)
    \end{equation*}
    en het potentiaalverschil van de bron als
    \begin{equation*}
        \Delta V = \Delta V_{\max}\sin(\omega t)
    \end{equation*}
    waarbij $\Delta V_{\max}$ de amplitude is van het potentiaalverschil. De stroom en het potentiaalverschil zijn in fase met elkaar.
\end{theo}

\begin{pro}[Weerstand in een wisselstroomkring]{Weerstand in een wisselstroomkring}
    Stel je een circuit voor met een AC-bron en een weerstand $R$. De tweede wet van Kirchhoff geeft
    \begin{equation*}
        \Delta v - \Delta v_{R} = \Delta v - i_R R = 0
    \end{equation*}
    de spanningsgroottes zijn dus gelijk en volgt dus dat
    \begin{equation*}
        \Delta v_{R} = \Delta V_{\max}\sin(\omega t) = I_{\max}R\sin(\omega t)
    \end{equation*}
    waarmee we de stroom door de weerstand kunnen bepalen in functie van de tijd
    \begin{equation*}
        i_R = \dfrac{\Delta v_{R}}{R} = \dfrac{\Delta V_{\max}}{R}\sin(\omega t) = I_{\max}\sin(\omega t).
    \end{equation*}
    \vspace{-0.5cm}
\end{pro}

\newpage

\begin{pro}[Zelfinductie in een wisselstroomkring]{Zelfinductie in een wisselstroomkring}
    Stel je een circuit voor met een AC-bron en een inductor met zelfinductie $L$. De tweede wet van Kirchhoff geeft
    \begin{equation*}
        \Delta v + \Delta v_L = \Delta v - \mathcal{E}_{L} = \Delta V_{0}\sin(\omega t) - L\dfrac{di}{dt} = 0
    \end{equation*}
    waaruit we een formule kunnen halen voor de stroom door de inductor
    \begin{equation*}
        i_L = \dfrac{\Delta V_{\max}}{L}\int\sin(\omega t)dt = -\dfrac{\Delta V_{\max}}{\omega L}\cos(\omega t)
    \end{equation*}
    wat we herleiden tot 
    \begin{equation*}
        i_L = \dfrac{\Delta V_{\max}}{\omega L}\sin(\omega t - \tfrac{\pi}{2}) = I_{\max}\sin(\omega t - \tfrac{\pi}{2})
    \end{equation*}
    door gebruik te maken van de goniometrisch identiteiten 
    \begin{equation*}
        \cos(\theta - \tfrac{\pi}{2}) = \sin(\theta) \quad \text{en} \quad \sin(-\theta) = -\sin(\theta).
    \end{equation*}
    We zien dat er een faseverschil is tussen de stroom en het potentiaalverschil van $90^{\circ}$, waarbij de stroom achter loopt.
\end{pro}

\begin{pro}[Capaciteit in een wisselstroomkring]{Capaciteit in een wisselstroomkring}
    Stel je een circuit voor met een AC-bron en een condensator met capaciteit $C$. De tweede wet van Kirchhoff geeft
    \begin{equation*}
        \Delta v - \Delta v_{C} = \Delta V_{\max}\sin(\omega t) - \dfrac{q}{C} = 0
    \end{equation*}
    waaruit we een formule kunnen halen voor de lading op de condensator
    \begin{equation*}
        q = C\Delta V_{\max}\sin(\omega t)
    \end{equation*}
    wat we herleiden tot een formule voor de stroom door de condensator
    \begin{equation*}
        i_C = \dfrac{dq}{dt} = \omega C\Delta V_{\max}\cos(\omega t) = I_{\max}\cos(\omega t)
    \end{equation*}
    wat we herleiden tot
    \begin{equation*}
        i_C = \omega C\Delta V_{\max}\sin(\omega t + \tfrac{\pi}{2}) = I_{\max}\sin(\omega t + \tfrac{\pi}{2})
    \end{equation*}
    door gebruik te maken van de goniometrische identiteiten
    \begin{equation*}
        \cos(\theta + \tfrac{\pi}{2}) = \sin(-\theta) \quad \text{en} \quad \cos(-\theta) = \cos(\theta).
    \end{equation*}
    We zien dat er een faseverschil is tussen de stroom en het potentiaalverschil van $90^{\circ}$, waarbij de stroom voor loopt.
\end{pro}

\newpage

% \begin{summ}[Toepassingen in wisselstroomkringen]{Toepassingen in wisselstroomkringen}
%     \def\arraystretch{2}
%     \centering
%     \begin{tabular}{c|c|c|c}
%         & $\Delta v$ & $i$ & $ X = \frac{\Delta V_{\max}}{I_{\max}}$ \\ \hline
%         $R$ & $\Delta V_{\max}\sin(\omega t)$ & $I_{\max}\sin(\omega t)$ & $X_{R} = R$ \\ [-0.35cm]
%         & $I_{\max}R\sin(\omega t)$ & $i$ in fase met $\Delta v$ & \\ \hline
%         $L$ & $\Delta V_{\max}\sin(\omega t)$ & $I_{\max}\sin(\omega t - \tfrac{\pi}{2})$ & $X_{L} = \omega L$ \\ [-0.35cm]
%         & $I_{\max}R\sin(\omega t)$ & $i$ loopt $90^\circ$ achter $\Delta v$ & \\ \hline
%         $C$ & $\Delta V_{\max}\sin(\omega t)$ & $I_{\max}\sin(\omega t +\tfrac{\pi}{2})$ & $X_{C} = \tfrac{1}{\omega C}$ \\ [-0.35cm]
%         & $I_{\max}R\sin(\omega t)$ & $i$ loopt $90^\circ$ voor $\Delta v$ & \\ 
%     \end{tabular}
% \end{summ}

\begin{pro}[Vermogen in een wisseltroomkring]{Vermogen in een wisseltroomkring}
    We kunnen het ogenblikkelijke vermogen $p$ geleverd door een AC-bron berekenen met
    \begin{equation*}
        p = i \Delta v = I_{\max}\Delta V_{\max}\sin(\omega t)\sin(\omega t - \phi) 
    \end{equation*} 
    waarop we de goniometrische identiteit $\sin(\omega t - \phi) = \sin(\omega t)\cos(\phi) - \cos(\omega t)\sin(\phi)$ kunnen toepassen 
    \begin{equation*}
        p = I_{\max}\Delta V_{\max}\left( \sin^2(\omega t)\cos(\phi) - \sin(\omega t)\cos(\omega t)\sin(\phi)\right).
    \end{equation*}
    Als we naar het gemiddelde kijken, dan zien we dat
    \begin{equation*}
        \sin^2(\omega t) = \tfrac{1}{2} \text{ en } \sin(\omega t)\cos(\omega t) = 0 
    \end{equation*} 
    waaruit we het gemiddelde vermogen kunnen afleiden
    \begin{equation*}
        P_{\text{gem}} = \frac{1}{2}I_{\max}\Delta V_{\max}\cos(\phi).
    \end{equation*}
    \vspace{-0.5cm}
\end{pro}

\begin{theo}[Inductieve en capacitieve reactantie]{Inductieve en capacitieve reactantie}
    \begin{itemize}
        \item   
        De \textbf{inductieve reactantie} $X_{L}$ van een spoel met zelfinductie $L$ is gedefinieerd als
        \begin{equation*}
            X_{L} = \omega L
        \end{equation*}
        waarbij $\omega$ de hoekfrequentie is van de wisselstroom. We zien 
        \begin{itemize}
            \item $ \omega \to 0: X_{L} \to 0 \quad \text{(kortsluiting)} $
            \item $ \omega \to \infty: X_{L} \to \infty \quad \text{(open circuit)} $
        \end{itemize}
        We kunnen nu ook de formule herwerken tot
        \begin{equation*}
            \Delta v_{L} = -I_{\max}X_{L}\sin(\omega t)
        \end{equation*}

        \item    
        De \textbf{capacitieve reactantie} $X_{C}$ van een condensator met capaciteit $C$ is gedefinieerd als
        \begin{equation*}
            X_{C} = \dfrac{1}{\omega C}
        \end{equation*}
        waarbij $\omega$ de hoekfrequentie is van de wisselstroom. We zien
        \begin{itemize}
            \item $ \omega \to 0: X_{C} \to \infty \quad \text{(open circuit)} $
            \item $ \omega \to \infty: X_{C} \to 0 \quad \text{(kortsluiting)}$
        \end{itemize}
            We kunnen nu ook de formule herwerken tot
        \begin{equation*}
            \Delta v_{C} = I_{\max}X_{C}\sin(\omega t)
        \end{equation*}
        
    \end{itemize}
    \vspace{0.3cm}
    \noindent \textbf{Opmerking:} Reactantie is een maat van weerstand tegen de ladingstroom, de reactantie van een weerstand $X_{R}$ is gewoon de weerstand $R$ zelf.
\end{theo}

\newpage

\begin{theo}[Impedantie van een RLC-kring]{Impedantie van een RLC-kring}
    \begin{minipage}{.68\textwidth}
        In de een serieschakeling is de stroom gelijk in alle componenten, bij een RLC-kring is dat
        \begin{equation*}
            i = I_{\max}\sin(\omega t - \phi)
        \end{equation*}
        waarbij $\phi$ de fasehoek is. De spanning, daarentegen, niet gelijk in alle component. We kunnen de spanningen op de componenten berekenen met behulp van de wet van Ohm
        \begin{equation*}
            \Delta v = \Delta v_{R} + \Delta v_{L} + \Delta v_{C} = 0
        \end{equation*}
    \end{minipage}
    \hspace{0.1cm}\vspace{-0.1cm}
    \begin{minipage}{.28\textwidth}
        \includegraphics[scale = 0.2]{Images/Magnetisme/RLCWisselstroomkring}
    \end{minipage}

    \vspace{0.3cm}
    \noindent waarbij we de spanningen kunnen berekenen met behulp van de formules voor de reactanties. We weten dat
    \begin{equation*}
        \forall i \in \{R,L,C\}: \ \Delta V_i = I_{\text{max}}X_i
    \end{equation*}
    wat de grootte bepaald in de fasediagrammen, we zullen nu deze diagrammen analyseren

    \begin{minipage}{.69\textwidth}
        \vspace{0.4cm}
        \hspace{-0,75cm}\includegraphics[scale = 0.48]{Images/Magnetisme/PotentialenFasoren.png}
    \end{minipage}
    \hspace{-0.3cm}$\longrightarrow$
    \begin{minipage}{.31\textwidth}
        \vspace{0.1cm}
        \includegraphics[scale = 0.3]{Images/Magnetisme/PotentiaalFasoren.png}
    \end{minipage}

    \vspace{0.5cm}
    \noindent hieruit kunnen we een formule afleiden voor $\Delta V_{\max}$ met behulp van de stelling van pythagoras, namelijk
    \begin{align*}
        \Delta V_{\max} &= \sqrt{(\Delta V_{R})^2 + (\Delta V_{L} - \Delta V_{C})^2} \\
                        &= I_{\max}\sqrt{R^2 + (X_{L} - X_{C})^2}.
    \end{align*}
    % wat we kunnen omvormen tot 
    % \begin{equation*}
    %     \Delta V_{\max} = I_{\max}\sqrt{R^2 + (X_{L} - X_{C})^2}.
    % \end{equation*}
    De \textbf{impedantie} $Z$ van de kring is gedefinieerd als
    \begin{equation*}
        Z = \dfrac{\Delta V_{\max}}{I_{\max}} = \sqrt{R^2 + (X_{L} - X_{C})^2}
    \end{equation*}
    wat de weerstand is van de kring tegen de wisselstroom. De fasehoek kunnen we ook berekenen met
    \begin{equation*}
        \phi = \tan^{-1}\left(\frac{X_L - X_C}{R}\right)
    \end{equation*} 
    wat triviaal te zien is. We kunnen bepaalde gevallen onderscheiden
    \begin{itemize}
        \item $X_{L} < X_{C}$: meer capacitief dan inductief 
        \item $X_{L} = X_{C}$: zuiver resisitief
        \item $X_{L} > X_{C}$: meer inductief dan capacitief
    \end{itemize}
\end{theo}

\newpage



