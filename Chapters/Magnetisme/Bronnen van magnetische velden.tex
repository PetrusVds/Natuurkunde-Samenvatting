\begin{theo}[Magnetisch veld ten gevolge van een rechte draad]{Magnetisch veld ten gevolge van een rechte draad}
    \begin{minipage}{0.87\textwidth}
        Het magnetische veld ten gevolge van de elektrische stroom in een lange rechte draad is zodanig dat de
        veldlijnen cirkels zijn met de draad in het midden (zie figuur). De veld sterkte is groter hoe dichter je bij de
        draad bent en hoe groter de stroom, in formulevorm:
        \begin{equation*}
            B \propto \dfrac{I}{r}
        \end{equation*}
        met r de loodrechte afstand van de draad. Deze relatie blijft waar als we aanemen dat de draad lang is.
        Het magnetisch veld nabij een lange, rechte draad is als volgt
        \begin{equation*}
            B = \dfrac{\mu_0}{2\pi}\dfrac{I}{r}
        \end{equation*}
        met $\mu_0$ de magnetische permeabiliteit in vacuum.
    \end{minipage}
    \begin{minipage}{.09\textwidth}
        \includegraphics[scale=0.225]{Images/Magnetisme/MagnetischVeldTenGevolgeRechteDraad}
    \end{minipage}
\end{theo}

\begin{app}[Magnetische kracht tussen twee parallelle draden]{Magnetische kracht tussen twee parallelle draden}
    \begin{minipage}{0.77\textwidth}
        Neem twee lange evenwijdige draden gescheiden door een afstand d, zoals in de figuur. Ze voeren respectievelijk stromen $I_1$ en $I_2$.
        Elke stroom produceert een magnetisch veld dat door de ander wordt ‘gevoeld’, dus oefenen ze een kracht uit op mekander. We weten dat
        \begin{equation*}
            F_{\text{max}} = I\ell B
        \end{equation*}
        en dus krijgen we:
        \begin{equation*}
            F = I_{2}B_{1}\ell_{2} = \dfrac{\mu_0}{2\pi}\dfrac{I_{1}I_{2}}{d}\ell_{2}
        \end{equation*}
    \end{minipage}
    \begin{minipage}{.19\textwidth}
        \includegraphics[scale=0.225]{Images/Magnetisme/MagnetischeKrachtTussenTweeParallelleDraden}
    \end{minipage} \vspace{0.2cm}\\
    Hieruit kunnen we afleiden dat evenwijdige stroomvoerende geleiders elkaar
    \begin{itemize}
        \item aantrekken indien de stroom in dezelfde richting vloeit,
        \item afstoten indien de stroom in tegengestelde richting vloeit.
    \end{itemize}
\end{app}

\begin{lem}[Ampère]{Ampère}
    \vspace{-0.75cm}\begin{minipage}{0.79\textwidth}
        De lijnintegraal van het magneetveld langs een gesloten pad is gelijk aan $\mu_{0}I_{\text{in}}$, waarbij $I_{\text{in}}$ de totale stroom is, die vloeit door een oppervlak dat omsloten is door het pad, in formulevorm:
        \begin{equation*}
            \oint \Vec{B} \cdot d\Vec{\ell} = \mu_{0}I_{\text{in}}
        \end{equation*}
    \end{minipage}
    \begin{minipage}{.17\textwidth}
        \includegraphics[scale=0.3]{Images/Magnetisme/WetVanAmpere}
    \end{minipage}
\end{lem}

\newpage

\begin{theo}[Het magnetisch veld van een spoel]{Het magnetisch veld van een spoel}
    Een lange draad met meerdere windingen noemen we een \textbf{spoel}, zie de figuren hieronder.

%    \vspace{0.5cm}

%    \begin{minipage}{.48\textwidth}
        \begin{center}
%            Spoel: \\
            \includegraphics[scale = 0.4]{Images/Magnetisme/Spoel}
        \end{center}
%    \end{minipage}
%    \begin{minipage}{.48\textwidth}
%        \begin{center}
%            \vspace{0.75cm}
%            Dicht gepakte spoel: \\
%            \includegraphics[scale = 0.25]{Images/Magnetisme/SpoelDicht}
%        \end{center}
%    \end{minipage}

%    \vspace{0.5cm}

    \noindent Elke winding genereert een magnetisch veld. Nabij elke draad zijn de veldlijnen quasi cirkels,
    net zoals bij een rechte draad. In het centrum van de spoel is het netto magnetisch veld een redelijk
    groot en redelijk uniform veld, we nemen aan dat een spoel dicht gebonden is en dus dat het veld praltisch uniform is.
    We kunnen de wet van Ampère toepassen op de zijden van de rechthoek (zie figuur onderaan)
    \begin{align*}
%        \hspace{1cm}
        \oint \Vec{B} \cdot d\Vec{\ell} &= \int_{a}^{b} \Vec{B} \cdot d\Vec{\ell} + \int_{b}^{c} \Vec{B} \cdot d\Vec{\ell}
        +  \int_{c}^{d} \Vec{B} \cdot d\Vec{\ell} +  \int_{d}^{a} \Vec{B} \cdot d\Vec{\ell} \\
                                        &= \int_{b}^{c} \Vec{B} \cdot d\Vec{\ell}  \\
                                        &= B\ell
    \end{align*}
    waarbij de integralen op de paden $a \to b$, $b \to c$ en $d \to a$ nul zijn, want het veld is praktisch nul tussen in en buiten de spoel.
    We berkenen nu de formule van het magnetisch veld van een spoel:
    \begin{align*}
        \oint \Vec{B} \cdot d\Vec{\ell} &= \mu_{0}NI \\
        B\ell &=  \mu_{0}NI \\
        B &=  \mu_{9}nI
    \end{align*}
    met $n= \tfrac{N}{\ell}$ het aantal windingen per lengte.
    \begin{center}
        \includegraphics[scale = 0.3]{Images/Magnetisme/SpoelMagnetischVeld}
    \end{center}
\end{theo}