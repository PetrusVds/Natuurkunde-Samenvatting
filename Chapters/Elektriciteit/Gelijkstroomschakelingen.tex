\begin{theo}[Elektromotorische “kracht”]{Elektromotorische “kracht”}
    Om stroom te hebben in een circuit, hebben we een apparaat nodig dat elektrische energie kan uitgeven. Dit apparaat wordt een bron van \textbf{elektromotorische kracht} (emf) genoemd.
    Het potentiaalverschil tussen de terminalen, ofwel de klemspanning, van de bron, wanneer er geen stroom vloeit, noemt men het emf $\mathcal{E}$ van de bron. Als er nu een stroom vloeit, dan is er een interne verzwakking van de klemspanning met een mate $Ir$ waarbij $r$ de interne weerstand is. In formulevorm krijgen we:
    \begin{equation*}
        \Delta V = \mathcal{E} - Ir
    \end{equation*}

    \begin{center}
        
        \begin{minipage}{.3 \textwidth}
            \includegraphics[scale = 0.45]{Images/Elektriciteit/Source.png}
        \end{minipage}
        \begin{minipage}{.3 \textwidth}
            \includegraphics[scale = 0.35]{Images/Elektriciteit/EMFgrafiek.png}
        \end{minipage}

    \end{center}

    \vspace{-0.5cm}
    % \noindent \textbf{Opmerking:} de kracht in elektromotorische kracht staat niet voor een newtoniaanse kracht, het betekent eerder dat het een drijfveer is voor de elektrishe stroom.
\end{theo}

\begin{pro}[Weerstanden bij gelijkstroomschakelingen]{Weerstanden bij gelijkstroomschakelingen}
    \begin{center}
        \def\arraystretch{1.5}
        \begin{tabular}{c|c}
             Parallel & Serie \\ \hline
             $ I = \sum_i I_i $ & $ I = I_i$ \\
             $ \dfrac{1}{R} = \sum_i \dfrac{1}{R_i} $ & $R = \sum_i R_i$ \\
             $ \Delta V = \Delta V_i $ &  $ \Delta V = \sum_i \Delta V_i $
        \end{tabular}
    \end{center}
\end{pro}

% \begin{pro}[Weerstanden in serie bij gelijkstroomschakelingen]{Weerstanden in serie}
%     \begin{itemize}
%         \item $ I = I_i $
%         \item $ R = \sum_i R_i $
%         \item $ \Delta V = \sum_i \Delta V_i $
%     \end{itemize}
% \end{pro}

% \begin{pro}[Weerstanden in parallel bij gelijkstroomschakelingen]{Weerstanden in parallel}
%     \begin{itemize}
%         \item $ I = \sum_i I_i $
%         \item $ \dfrac{1}{R} = \sum_i \dfrac{1}{R_i} $
%         \item $ \Delta V = \Delta V_i $
%     \end{itemize}
% \end{pro}

\begin{theo}[Eerste regel van Kirchhoff]{Eerste regel van Kirchhoff}
     De som van de stromen die een vertakking binnenkomen, moet gelijk zijn aan de som van de stromen die de vertakking verlaten. In formulevorm:
     \begin{equation*}
         \sum I_{in} = \sum I_{uit}
     \end{equation*}
     \vspace{-0.5cm}
\end{theo}

\begin{theo}[Tweede regel van Kirchhoff]{Tweede regel van Kirchhoff}
    De som van de potentiaalverschillen over alle elementen in een gesloten kring, moet nul zijn. In formulevorm:
    \begin{equation*}
        \sum_{\text{gesloten kring}} \Delta V = 0
    \end{equation*}
    \vspace{-0.4cm}
\end{theo}

\newpage

\begin{app}[RC-kringen]{RC-kringen}
    \vspace{-0.3cm}
    \begin{minipage}{.73\textwidth}
        Als de schakelaar dicht is, dan laadt de condensator op tot het het potentiaalverschil heeft van de batterij.
        We kunnen hierop de tweede regel van Kirchhoff toepassen
        \begin{equation*}
            \mathcal{E} - \dfrac{q}{C} - IR = 0
        \end{equation*}
        waaruit volgt:
    \end{minipage}
    \begin{minipage}{.25\textwidth}
       \includegraphics[scale = 0.15]{Images/Elektriciteit/RC-kring.png}
    \end{minipage}
    \vspace{-0.5cm}
    \begin{equation*}
        \mathcal{E} - \dfrac{q}{C} - \frac{dq}{dt}R = 0
    \end{equation*}
    % \begin{align*}
    %     t &= 0: \mathcal{E} - I_0R = 0 \\ 
    %     t &= \infty: \mathcal{E} - \dfrac{Q}{C} = 0.
    % \end{align*}
    \noindent Uit deze vergelijking leiden we de volgende formules voor de stroom en de ogenblikkelijke lading op de condensator af
    \begin{equation*}
        q = C\mathcal{E}(1-e^{-\tfrac{t}{\tau}})= Q(1-e^{-\tfrac{t}{\tau}}) \ \Rightarrow \ I = \dfrac{dq}{dt} = \dfrac{\mathcal{E}}{R}e^{-\tfrac{t}{\tau}}
    \end{equation*}

    % \begin{equation*}
    %      I = \dfrac{dq}{dt} = \dfrac{d}{dt} C\mathcal{E}(1-e^{-\tfrac{t}{RC}}) = \dfrac{\mathcal{E}}{R}e^{-\tfrac{t}{RC}}
    % \end{equation*}
    % \newpage
    
    \noindent waarbij $\tau = RC$ de \textbf{tijdsconstante} die de tijd voorstelt nodig voor een condensator om tot $ 63\%$ van zijn lading en voltage te bekomen: $RC$ is een maateenheid voor de snelheid van het opladen van de condensator. \\
    \vspace{-0.2cm}

    \begin{minipage}{0.48\textwidth}
        \vspace{0.21cm}\hspace{1.4cm}\includegraphics[scale = 0.195]{Images/Elektriciteit/Tijdsconstante1.png}
    \end{minipage}
    \begin{minipage}{0.48\textwidth}
        \includegraphics[scale = 0.23]{Images/Elektriciteit/Tijdsconstante2.png}
    \end{minipage}
    
    \noindent We kunnen nu de volgende formules opstellen voor de energie in de RC-kring:
    \begin{itemize}
        \item Hoeveel energie heeft de emf geleverd? 
        \begin{equation*}
            U_{\text{emf}} = \int_0^Q \mathcal{E} dq = Q\mathcal{E} = \mathcal{E}^2C
        \end{equation*}
        \item Hoeveel energie is er opgeslagen in de condensator?
        \begin{equation*}
            U_{C} = \dfrac{\mathcal{E}^2C}{2} = \dfrac{Q^2}{2C}
        \end{equation*}
        \item Hoeveel warmte is er gedisipeerd in de weerstand?
        \begin{equation*}
            U_{R} = \dfrac{\mathcal{E}^2C}{2} = \dfrac{Q^2}{2C}
        \end{equation*}
    \end{itemize}
    We zien dus dat de energie geleverd door de emf netjes is verdeeld over de weerstand en de condensator, namelijk:
    \begin{equation*}
        U_{\text{emf}} = U_{C} + U_{R}
    \end{equation*}
\end{app}

