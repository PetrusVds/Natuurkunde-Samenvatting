\begin{theo}[Impulsmoment]{Impulsmoment}
     Het \textbf{Impulsmoment} is de rotationele variant van het translatonele impuls, net zoals krachtmoment rotationele variant is van de translationele kracht. Het wordt gegeven door de volgende formule:
    
     \begin{equation*}
         \Vec{L} = \Vec{r} \times \Vec{p}
     \end{equation*}

     \noindent De grootte van het impulsmoment kunnen we als volgt berekenen:
     \begin{equation*}
         L = prsin(\phi) = mvrsin(\phi) = m(\omega r)rsin(\phi) = I\omega
     \end{equation*}
     \vspace{-0.5cm}
\end{theo}

\begin{app}[Verband tussen impulsmoment en krachtmoment]{Verband tussen impulsmoment en krachtmoment}

    Net zoals kracht zorgt voor de verandering van impuls van een voorwerp, zorgt krachtmoment voor de verandering van impulsmoments van een voorwerp: dit is het voornaamste verband. Als we de tweede wet van newton bekijken bij rotatie, dan krijgen we de volgende formule die het verband aantoont:
    
    \begin{equation*}
        \sum \Vec{\tau} = I\Vec{\alpha} = I(\dfrac{d\Vec{\omega}}{dt}) = \dfrac{d(I\Vec{\omega})}{dt} = \dfrac{d\Vec{L}}{dt}
    \end{equation*}

    \noindent Omgekeerd kan ook als we een \textbf{puntmassa} bekijken:
    \begin{equation*}
        \dfrac{d\Vec{L}}{dt} = \dfrac{d(\Vec{r} \times \Vec{p})}{dt} = \dfrac{d\Vec{r}}{dt} \times \Vec{p} + \Vec{r} \times \dfrac{d\Vec{p}}{dt} = \Vec{v} \times \Vec{p} + \Vec{r} \times \Vec{F}
    \end{equation*}
    \noindent Het eerste lid van de som, namelijk $ \Vec{v} \times \Vec{p} $, wordt simpelweg 0, $ \Vec{v} $ is een lineaire combinatie van $ \Vec{p} $. Nu hebben we:
    \begin{equation*}
        \dfrac{d\Vec{L}}{dt} = \Vec{r} \times \Vec{F}
    \end{equation*}
    \noindent Als we nu $ \sum \Vec{F} $ pakken als de netto kracht, dan is bij een inertiaalreferentiestelsel:
    \begin{equation*}
       \dfrac{d\Vec{L}}{dt} = \Vec{r} \times \sum \Vec{F} = \sum \tau
    \end{equation*}
    \vspace{-0.5cm}
\end{app}

\begin{app}[Vergelijking: impuls – impulsmoment]{Vergelijking: impuls – impulsmoment}
    \begin{minipage}{.4\textwidth}
        \begin{center}
            
            \underline{Algemeen}: \\
            \vspace{0.25cm}
            \def\arraystretch{2.5}
            \begin{tabular}{c|c}
                impuls & impulsmoment \\ \hline
                $ \Vec{p} $ & $ \Vec{L} = \Vec{r} \times \Vec{p} $ \\ 
                $ \Vec{F} = \dfrac{d\Vec{p}}{dt} $ &  $ \Vec{\tau} = \dfrac{d\Vec{L}}{dt} $
            \end{tabular}
    
        \end{center}
    \end{minipage} 
    \begin{minipage}{.6\textwidth}
        \begin{center}
                
            \underline{Enkel bij een geïsoleerd systeem}: \\
            \vspace{0.25cm}
            \def\arraystretch{2.5}
            \begin{tabular}{c|c}
                impuls & impulsmoment \\ \hline
                behoud van impuls & behoud van impulsmoment \\ 
                $ F_{net} = \sum \Vec{F}_{ext} = 0 $ & $ \tau_{net} = \sum \Vec{\tau}_{ext} = 0 $
            \end{tabular}
        
        \end{center}
    \end{minipage}
\end{app}

